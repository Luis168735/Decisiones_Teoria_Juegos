\documentclass[12pt]{scrartcl} % se puede cambiar por article
\RequirePackage{amsmath,amsfonts,amssymb,amsthm}
\usepackage{eulervm} % tipografía con soporte para matemáticas
\usepackage{caption}
\usepackage{exercise}

\usepackage[utf8]{inputenc}
\usepackage[spanish,mexico]{babel}
% sgamex.sty from Rubinstein
\usepackage{sgamex}
\usepackage{dtj_iteso} % creación mía

\usepackage{setspace}
\onehalfspacing % doble espacio: \doublespacing

\title{Examen de Juan Charrasqueado}
\author{Juan Charrasqueado}
\date{\today}

\begin{document}
\maketitle

Blabla mis respuestas son las mejores blabla

% El paquete sgamex permite hacer tablas como esta
\begin{center} % para centrar sin tanto rollo
	\begin{game}{2}{2}[Trabajador 1][Trabajador 2]
					&               Aplicar E1         & Aplicar E2 \\
		Aplicar E1  & $\frac{1}{2}w_1, \frac{1}{2}w_1$ & $w_1, w_2$ \\ 
		Aplicar E2  & $w_2, w_1$                       & $\frac{1}{2}w_2, \frac{1}{2}w_2$
	\end{game}
\end{center}

% Respuestas enumeradas con paquete Exercise

\begin{Exercise}[name={Respuesta}] 
% el número de la respuesta se actualiza cada que es usado
% \begin{Exercise}

	\[
		q^* = \argmax_{0 \leq q < \infty} %operador argmax
	\]
\end{Exercise}

% Ejemplo de un árbol de decisión

\begin{figure}[H]
	\centering
	\footnotesize{
		\begin{forest} decision tree,for tree={s sep=25pt}
			[J1, plain content
				[{2,2};L,plain content,elo={yshift=4pt}]
				[J2;R,plain content,elo={yshift=4pt}
					[{3,1}; l, plain content,elo={yshift=4pt}]
					[J1;r,plain content,elo={yshift=4pt}
						[J2;A,plain content,elo={yshift=4pt},alias=j2i
							[{2,-2};c]
							[{-2,2};d]
						]
						[J2;B,plain content,elo={yshift=4pt},alias=j2d
							[{-2,2};c]
							[{2,-2};d]
						]
					]
				]
			]
			% esta línea dibuja una línea punteada entre los nodos de
			% J2, usamos alias j2i (izquierda) j2d (derecha)
			\draw[dashed,transform canvas={yshift=-6pt}] (j2i) to[right=45] (j2d);
		\end{forest}}
	\caption{Forma extensiva de un juego}
	\label{fig:fig1}
\end{figure}

\end{document}