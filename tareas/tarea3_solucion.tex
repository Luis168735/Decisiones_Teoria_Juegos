\documentclass[12pt]{scrartcl} % se puede cambiar por article
\RequirePackage{amsmath,amsfonts,amssymb,amsthm}
\usepackage{eulervm} % tipografía con soporte para matemáticas
\usepackage{caption}

\usepackage{exercise}
\usepackage{graphicx}
\DeclareMathOperator*{\argmax}{argmax}

\usepackage[utf8]{inputenc}
\usepackage[spanish,mexico]{babel}
% sgamex.sty from Rubinstein
\usepackage{sgamex}

\usepackage{setspace}
\onehalfspacing % doble espacio: \doublespacing

\title{Solución de la tarea 3}
\author{Emmanuel Alcalá}
\date{\today}

\begin{document}
\maketitle

% Respuestas enumeradas con paquete Exercise
\begin{Exercise}[name={Respuesta}]

Las valuaciones tienen distribución
$ x_1, x_2, x_2 \sim \text{uniforme}(0, 30) $
Los jugadores 2 y 3 pujan según

\begin{align*}
  b_2(x_2) &= \frac{3}{4}x_2\\
  b_3(x_3) &= \frac{4}{5}x_3
\end{align*}

Jugador 1 gana si $ b_1 > b_2 $ \textbf{y} si $ b_1 > b_3$. Por lo tanto, la probabilidad de ganar es 

\[p(\text{ganar}) = p( b_1 > b_2 ) \times p ( b_1 > b_3 )\]

es decir

\begin{align*}
  p(\text{ganar}) &= p \left( x_2 < \frac{4}{3}b1 \right) \times p \left( x_3 < \frac{5}{4}b1 \right)\\
  \text{que es} & \text{ a su vez}\\
  p(\text{ganar}) &= \frac{4}{3}\times\frac{b_1}{30}\times\frac{5}{4}\times\frac{b_1}{30}\\
  p(\text{ganar}) &= \frac{b_1^2}{540}
\end{align*}

La utilidad esperada del jugador 1 si gana es 

\[UE_1(b_1, b_2, b_3) = p(\text{ganar})(x_1 - b_1)\]

Sustituyendo $ p(\text{ganar}) $ tenemos

\[UE_1(b_1, b_2, b_3) = \frac{1}{540}b_1^2(x_1 - b_1)\]

dado que la condición de optimalidad es

\[b_1^* = \argmax_{b_1 \in [0, x_1]} UE_1(b_1, b_2, b_3)\]

se satisface con la CPO:

\begin{align*}
  \frac{\partial UE_1(b_1, b_2, b_3)}{\partial b_1} &= 0\\
  \frac{1}{540} \frac{\partial (b_1^2x_1 - b_1^3)}{\partial b_1} &=0\\
   2b_1x_1 - 3b_1^2 &= 0\\
   b_1(2x_1 - 3b_1) &= 0\\
\end{align*}

del cual resultan dos soluciones:

\begin{enumerate}
  \item $b_1 = 0$
  \item $b_2 = \frac{2}{3}x_1$
\end{enumerate}

\end{Exercise}

\begin{Exercise}[name={Respuesta}]

  \begin{enumerate}
    \item $ a = \frac{2}{3} $
    \item El valor de $ a $ en equilibrio depende solo del número de jugadores, y no de sus pujas.
  \end{enumerate}

\end{Exercise}

\begin{Exercise}[name={Respuesta}]

De la solución en clase de Subasta de sobre cerrado al primer precio con aversión al riesgo, con la función $ u(w) = w^\alpha $, llegamos a la siguiente solución
    \[b_1^* = \frac{1}{1+\alpha}x_1\]
    con lo que concluimos que, si la estrategia óptima del jugador es $ b_1^* = ax_1$, entonces $ a = \frac{1}{1+\alpha} $, por lo que el valor de $ \alpha $ debe ser $ 0.5 $.

\end{Exercise}

\begin{Exercise}[name={Respuesta}]

  Los datos del problema son: $ P(Q) = 10 - Q\; \text{ para la empresa 2 } c_A = 4 \text{ con } p =1/2, c_B = 0  \text{ con } p =1/2, \text{ y para la empresa 1 } c = 0$.

  \begin{enumerate}
    \item La función de utilidad para el jugador 1 es 
    \[u_1(q_1; q_{2,A}, q_{2, B}) = \frac{1}{2}(10 - q_1 - q_{2,A})q_1 + \frac{1}{2}(10 - q_1 - q_{2,B})q_1\]
    Que simplificando queda
    \[u_1(q_1; q_{2,A}, q_{2, B}) = \left( 10 - q_1 - \frac{q_{2,B}}{2}- \frac{q_{2,A}}{2} \right)q_1\]
    La función de utilidad para el jugador 2 es
    \begin{align*}
      u_2(q_1, q_{2,t}) =%
      \begin{cases}
        (10 - q_1 - q_{2,B})q_{2,B} &\text{si } t = \text{B}\\
        (10 - q_1 - q_{2,A})q_{2,A} - 4q_{2,A} &\text{si } t = \text{A}
      \end{cases}
    \end{align*}
  
    \item Las condiciones de optimización son
    \begin{align*}
      q_1^*    & = \argmax_{0\leq q_1 \leq \infty} u_1(q_1; q_{2,A}, q_{2,B}) \\
      q_{2,A}^* & = \argmax_{0\leq q_{2,A} \leq \infty} u_2(q_1, q_{2,A})      \\
      q_{2,B}^* & = \argmax_{0\leq q_{2,B} \leq \infty} u_2(q_1, q_{2,B})
    \end{align*}
    y 
    \[\frac{\partial u_1}{\partial q_1}=0,\quad \frac{\partial u_{2,B}}{\partial q_{2,B}},\quad \text{y } \frac{\partial u_{2,A}}{\partial q_{2,A}} \]
    \item Para $ \frac{\partial u_1}{\partial q_1}=0 $ tenemos
    \[\frac{\partial u_1}{\partial q_1} = 10 - 2q_1 - \frac{1}{2}q_{2,B} - \frac{1}{2}q_{2,A}= 0\]
    Para $\frac{\partial u_{2,B}}{\partial q_{2,B}} = 0$ y $ \frac{\partial u_{2,A}}{\partial q_{2,A}} = 0 $
    \[\frac{\partial u_{2,B}}{\partial q_{2,B}} \longrightarrow q_{2,B}^* = \frac{10-q_1}{2} \]
    y 
    \[\frac{\partial u_{2,A}}{\partial q_{2,A}} \longrightarrow q_{2,A}^* = \frac{6-q_1}{2} \]
    Sustituyendo $  q_{2,A}^* \text{ y }  q_{2,B}^* \text{ en } q_1^*$, tenemos (luego de algunas manipulaciones):
    \[q_1^* = 4, q_{2,A}^* = 1 \text{ y } q_{2,B}^* = 3\]
    \item $ u_2(q_1, q_{2,A}) = (10-q_1-q_{2,A})q_{2,A} - 4q_{2,A} = (10-4-1)1-4=1$
  \end{enumerate}

\end{Exercise}

\begin{Exercise}[name={Respuesta}]

  Los datos del problema son: $ P(Q) = 1 - Q$ para la empresa 2 $c_A$ con probabilidad $\theta$, $c_B \text{ con probabilidad } 1-\theta$,y para la empresa 1 $c = 0$.

  Las funciones de utilidad son

  \begin{align*}
    u_1(q_1; q_{2,A}, q_{2, B}) &= \theta(1 - q_1 - q_{2,A})q_1 + (1-\theta)(1 - q_1 - q_{2,B})q_1\\
      u_2(q_1, q_{2,t}) &=%
      \begin{cases}
        (1 - q_1 - q_{2,B})q_{2,B} - c_Bq_{2,B}&\text{si } t = \text{B}\\
        (1 - q_1 - q_{2,A})q_{2,A} - c_Aq_{2,A} &\text{si } t = \text{A}
      \end{cases}
  \end{align*}

  \begin{enumerate}
    \item   Las condiciones de optimización son
    \begin{align*}
      q_1^*    & = \argmax_{0\leq q_1 \leq \infty} u_1(q_1; q_{2,A}, q_{2,B}) \\
      q_{2,A}^* & = \argmax_{0\leq q_{2,A} \leq \infty} u_2(q_1, q_{2,A})      \\
      q_{2,B}^* & = \argmax_{0\leq q_{2,B} \leq \infty} u_2(q_1, q_{2,B})
    \end{align*}
    y 
    \[\frac{\partial u_1}{\partial q_1}=0,\quad \frac{\partial u_{2,B}}{\partial q_{2,B}},\quad \text{y } \frac{\partial u_{2,A}}{\partial q_{2,A}} \]
    \item Resolviendo para $ \frac{\partial u_1}{\partial q_1}=0 $ nos da:
    \[q_1* = \frac{1 - q_{2,B}- \theta q_{2,A} + \theta q_{2,B}}{2}, \text{ equivalente a } q_1^* = \frac{1}{2} - \frac{(1-\theta)q_{2,B}-  \theta q_{2,A}}{2}\]
    Y para $ \frac{\partial u_{2,t}}{\partial q_{2,t}}=0 $
    \begin{align*}
      q_{2,t}^* =%
      \begin{cases}
        \frac{1-q_1-c_A}{2} \text{ si } t = A\\
        \frac{1-q_1-c_B}{2} \text{ si } t = B
      \end{cases}
    \end{align*}
    Sustituyendo $ q_{2,A}^*, q_{2,B}^*$ en $ q_1^* $, y haciendo manipulaciones algebraicas, nos retorna
    \begin{align*}
      q_1^* &= \frac{1+(1-\theta)c_B + \theta c_A}{3} \text{ y sustituyendo en } q_{2,A}^*, q_{2,B}^*\\
      \color{blue!70!cyan}q_{2,A}^* &{\color{blue!70!cyan}= \frac{2-(1-\theta)c_B - \theta c_A - 3 c_A}{6}} \text{ o equivalentemente}\\
      \color{blue!70!cyan}q_{2,A}^* &{\color{blue!70!cyan}= \frac{1-2c_A}{3} - \frac{(1-\theta)(c_B-c_A)}{6}} \\
      \color{red!60!black}q_{2,B}^* &{\color{red!60!black}= \frac{2-(1-\theta)c_B - \theta c_A  3c_B}{6}} \text{ o equivalentemente}\\
      \color{red!60!black}q_{2,B}^* & {\color{red!60!black} = \frac{1-2c_B}{3} + \frac{\theta(c_B-c_A)}{6}}
    \end{align*}
    \item $ \frac{\partial q_{2,A}^*}{\partial \theta} $ y $ \frac{\partial q_{2,B}^*}{\partial \theta}  $
    \begin{align*}
      \frac{\partial q_{2,A}^*}{\partial \theta} &= \frac{\partial}{\partial \theta} \left[ \frac{1-2c_A}{3} - \frac{(1-\theta)(c_B-c_A)}{6} \right] = \frac{1}{6}(c_B - c_A) \\
      \frac{\partial q_{2,B}^*}{\partial \theta} &= \frac{\partial}{\partial \theta} \left[\frac{1-2c_B}{3} + \frac{\theta(c_B-c_A)}{6}\right] = \frac{1}{6}(c_B - c_A)
    \end{align*}
    La cantidad en equilibro en ambos costos depende de la diferencia de los costos. Dado que $ c_A > c_B $ por definición, la cantidad en ambos casos decrece conforme $ \theta $ crece. 
  \end{enumerate}
  
\end{Exercise}

\end{document}

