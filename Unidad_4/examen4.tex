\PassOptionsToPackage{table}{xcolor}
\documentclass[12pt]{scrartcl} % se puede cambiar por article
\RequirePackage{amsmath,amsfonts,amssymb,amsthm}
\usepackage{eulervm} % tipografía con soporte para matemáticas
\usepackage{caption}
\usepackage{exercise}
\usepackage{tikzsymbols}

\usepackage[utf8]{inputenc}
\usepackage[spanish,mexico]{babel}
% sgamex.sty from Rubinstein
\usepackage{sgamex}
\usepackage{dtj_iteso} % creación mía

\usepackage{setspace}
\onehalfspacing % doble espacio: \doublespacing

\title{Examen - Unidad 4 \\ \normalsize Decisiones y Teoría de Juegos}
\author{Emmanuel Alcalá\\ \url{jaime.alcala@iteso.mx}}
\date{\today}


\begin{document}
\maketitle
\hrule

\begin{summarybox}{Instrucciones}

  \begin{description}
    \item[1] - Entregar el examen en la fecha acordada en CANVAS, \textbf{\color{blue} en formato pdf}. %Puedes contestar en papel, tomar fotos y colocarlas en un archivo Word y convertirlo a pdf, luego subirlo en la entrada de CANVAS correspondiente al examen, o como te parezca apropiado, pero en \textbf{pdf}.
    \item[2] - Escribe \textbf{claramente} los cálculos que desarrollaste para resolver los problemas, y \textbf{encierra en un recuadro} la respuesta correcta.
    \item[3] - {\color{blue} IMPORTANTE}: 1) problemas que solo contengan respuestas sin desarrollo los consideraré {\color{red} erróneos}; 2) copiar y plagiar respuestas no será tolerado, y cada respuesta identificada como copia de la respuesta de otro estudiante será {\color{red} anulada} para ambos estudiantes.
    \item[4] - No habrá plazos extras. Si existe una situación \textit{extraordinaria}, házmelo saber con tiempo y evidencia.
    \item[5] - Las únicas dudas que contestaré serán relativas a la redacción (errores o alguna confusión). %No responderé si un resultado es correcto, ni mucho menos cómo resolver algo. Los ejercicios han sido seleccionados estrictamente dentro del material revisado en clases y en las notas de clase. 
  \end{description}

\end{summarybox}

\begin{center}
  \Coffeecup[1.5]
\end{center}

% Respuestas enumeradas con paquete Exercise
\begin{Exercise}[title={Subasta de sobre cerrado con $n$ jugadores},name={Pregunta}]
  \textit{7.5 pt}

  Considera el siguiente juego de dos entre un empleador y un aplicante. El tipo del aplicante es su intelecto, que puede ser bajo (\textit{low}), medio (\textit{moderate}) o alto (\textit{high}), con probabilidades de $ 1/3, 1/2 $ y 1/6, respectivamente. 

  Luego de que el aplicante conozca su tipo, decide si ir o no a la universidad (\textit{college, no college} o para simplificar, \textit{C, NC}). El costo personal de ir a la universidad es mayor si el aplicante es menos inteligente.
  
  Asumir que el costo de obtener un grado es de 2, 4 y 6 para un aplicante cuyo tipo sea alto, mediano y bajo, respectivamente. 

  El empleador decide si ofrecer al aplicante un trabajo como administrador (\textit{manager}) u oficinista (\textit{clerk}). El pago del aplicante por ser contratado como administrador es de 15, y por oficinista de 10. Esos pagos son independientes del tipo del aplicante. 

  El empleador gana 7 si emplea a alguien como oficinista, \textit{independientemente} del tipo del aplicante. 

  Por otro lado, si el aplicante es contratado como administrador, las ganancias del empleador crecen acorde al intelecto del aplicante: de 4, 6 a 14, dependiento de si el intelecto es bajo, moderado o alto, respectivamente.

\begin{center}
  \includegraphics[width=\textwidth]{p1.png}
\end{center}

\textbf{Resuelve:}
(Respuestas \textit{completas})

\begin{enumerate}
  \setlength{\itemsep}{0pt}
  \setlength{\parskip}{0pt}
  \setlength{\parsep}{0pt}
  \item ¿De qué dependen las ganancias de cada jugador? (\textit{2pt})
  \item ¿Cuántos conjuntos de información existen? (\textit{2pt})
  \item Encuentra un EBP en donde los estudiantes de bajo intelecto no van a la universidad, y los de moderado y alto sí van (es decir, un EBP con la estrategia del estudiante $ \{NC_{low}, C_{moderate}, C_{high}\} $).
  \begin{itemize}
    \item ¿Cuáles son las creencias del empleador? (\textit{4pt})
    \item ¿Cuáles son las mejores respuestas del empleador? (\textit{4pt})
    \item Evalúa las posibles desviaciones del aplicante para todos sus tipos, y concluye si la estrategia considerada es un EBP (\textit{3pt})
  \end{itemize}
\end{enumerate}

\textbf{Pista:} Comienza con las creencias del empleador. Recuerda que actualiza sus creencias de acuerdo al teorema de Bayes. Por ejemplo, si \textit{observa} que el estudiante no fue al colegio, su creencia se computa según:

\begin{align*}
  p(\text{low}|NC) = \frac{p(NC | \text{low})p(\text{low})}{p(NC | \text{low})p(\text{low}) + p(NC | \text{moderate})p(\text{moderate})+p(NC | \text{hight})p(\text{hight})}
\end{align*}

Nota que $ p(NC | \text{low}) $ hace referencia a lo observado por el empleador según la estrategia del aplicante. Para esta estrategia, esa probabilidad es de 1 (ahora computa el resto). 

Luego evalúa las utilidades del empleador por contratar como administrador o como oficinista. Evalúa cada mensaje: si va a la universidad, y si no va a la universidad, usando $ UE_{Empleador}(\text{Manager}), UE_{Empleador}(\text{Clerk}) $.

Luego, evalúa por tipo de aplicante si le conviene desviarse de sus estrategias, \textit{dadas} las mejores respuestas del empleador.

\end{Exercise}

\begin{center}
  ¡Suerte!

  \LARGE\Cat[1.2]
\end{center}


\end{document}